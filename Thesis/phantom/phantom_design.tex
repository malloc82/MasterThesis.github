\section{Introduction}

MAGNETIC resonance imaging (MRI) provides an excellent modality for distinguishing different tissues in the 
human body, which is essential for medical applications. When MRI is used for stereotactic treatment planning,
its geometric accuracy is crucial. Previous studies have been conducted to correct the distortion caused by 
nonlinearity of the MRI scanner’s magnetic gradient fields by imaging a cubic phantom \cite{simple_approach},
\cite{tlee_iaeng} and defining 
distortion correction functions based on the distorted appearance its surfaces. Correct mathematical handling 
of the distortion correction function required that the cube was centered about the magnetic isocenter, which 
is defined as the common center of the three magnetic gradient fields and  is not very accurately known.  
New 3Tesla MRI systems have a larger bore, making the previous phantom design impractically small for probing 
the field nonlinearity in the periphery of the bore, as the phantom cannot be scaled up due to constraints 
related to weight and cost of manufacturing. We are introducing a new phantom design using a different 
approach, which can be built to a larger size, improves accuracy of distortion characterization and reduces 
cost.

\section{Design}

The phantom is shaped to be as large as possible, while still fitting in the head coil of the scanner, so as 
to achieve the maximum quality of signal and largest distortion. The phantom is an octagonal prism with 205mm 
between opposite sides.  Each face of the octagonal prism is composed of 8 high precision NMR tubes that are 
5mm in outer diameter and 205 mm in length. These tubes are filled with copper sulfate solution to generate a 
strong signal in an MRI scanner.  The tubes are placed parallel to the magnetic field so they will cause less 
susceptibility distortion \cite{mag_susceptibility}, 
and thus provide more accurate information on the gradient field.  In the 
center of the phantom is a large water tank to help intensify the signals generated by the tubes. At one end 
of the phantom is a cylindrical tank filled with copper sulfate, with a number of small solid cylinders in a 
hexagonal pattern that are connecting the two surfaces of the tank. These cylinders are used to maintain the 
long-term accuracy of the two surfaces, making sure they won’t deform, and are also aligned to the field to 
minimize their effect on the field.  The data generated from 64 tubes mounted on the sides are designed to 
give us x and y axis distortion information, and the end tank is designed to provide z-axis distortion data, 
allowing a complete 3-D distortion correction with a relatively small amount of data. 

\section{Initial Experience}

The original phantom design, a 16-cm oil-filled cube, could not be scaled up due to weight and manufacturing 
constraints. Our first modification was to look at changing the material to FR-4 since it is rigid, very flat, 
and could be submerged for short periods of time to permit scanning the solid liquid interface in an MRI 
machine.  Since most of the distortion is visible only in the corners, this was rapidly replaced by 8 corners 
of a virtual cube, which could be connected by a rigid frame.  The weight of a tank to submerge either of 
these designs to allow scanning was prohibitive, requiring a complete redesign.

\section{Conclusion}

A new phantom for the measurement and correction of nonlinear gradient field distortion of MRI systems is 
presented.  It is lighter, more economic, and allows for both more accurate measurement of field distortion 
and the location of the isocenter. 