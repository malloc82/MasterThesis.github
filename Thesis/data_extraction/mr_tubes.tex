\subsection{Introduction}
There are total 64 MR tubes filed Copper Sulfate $(CuSO_4)$ places on the outer region of the phantom.
We filled them with $CuSO_4$ due to this solution has very good magnetic suspetibility inside MRI scanner. 
They produce very strong signals in MR images which make them much easier to extract. Tubes can be purchased
separately, and easy to replace if necessary. Tubes are sealed with silicon to prevent leaking and could stored
vertically for a long period of time. 

These tubes are used to estimate MRI scanner's gradient isocenter as well as calculate the x and y axis' 
undistortion parameters. To calculation undistortion parameter, we need both original accurate 3D locations
and distorted 3D locations of each tubes. We can use tube holders' spacing specified in phantom's 
specification to determine their undistorted locations since these holders are precision made. So the missing
information are distorted tubes locations. MRI's distortion is nonlinear, which means the 3D shifts at
different locations are different. The way we extract tubes' distortion is by analyzing the axial 
scan/reconstruction of the phantom to determine a starting and ending slice of those tubes, and for every
slices in between them, we find the center of the each tube. Each bended tube in MR images can be viewed as
series. We will use these cetners in later calculation of undistortion parameters.

\subsection{Range}
This section describes how to determine the first and the last axial slices that are used to extract
tube centers.

Tube holders at both end of the MR tubes will cause the regions of the tubes inside the tube holder to shift. 
Depending on how the original scan is performed, the shifts' direction may vary. If the original scan is
done in axial directin, the the shift will occur inside xy plane; if the original scan is performed in
sagittal direction the shift will occur on z-axis. So this means that we cannot use the signal from the 
regions where tube holders are at. 

One way to determine where the tube holders's z-axis location is to find the middle water tank boundaries. 
Tube holder is usually at least 5mm away from each  tank surfaces in MR images. So after finding the location
of two water tank surfaces we can add or subtract 4mm as two boundaries. For every axial slices, we will
check it's z-axis location to see if that slice is between the boundary, if it is, we will continue to process
the slice, otherwise we will ignore it.

To find a accurate middle tank surfaces, we can work on either coronal or sagittal series, and use the
following steps:

\begin{enumerate}
\item Averge the middle slice of the series with its two adjacent slices to produce a sample slice. This 
  way we can get rid of some noise and would give a more reliable result.
\item In the sample slice, we average middle 3 columns of pixels to produce a sample column.
\item Produce a gradient for the sample column. 
\item The top surface of the middle water tank could be set to be the first peak in second half 
  of the data that is greater than 150; and the bottom surface of the water tank could be set to the 
  last peak in the first half that's lower than -150. 
\item Two pixels will be used for calculating the z-axis location of the two tank surfaces. Their row
  values are the two values we calculated from previous step repectively, and their column values are both
  middle column of the middle slice of this series.
\item Two pixels will be converted into 3D dicom coordinate using DICOM affine formula. The one with positive
  z coordinate will be added 4mm and the one with negative z coordinate will be subtract 4mm. These two 
  z coordinates defines a range which axial slices used to extract MR tubes centers should be in.
\item To validate a axial slice, we use the ImagePatientPosition parameter to see if the slice is in the range.
\end{enumerate}

\subsection{Tube Centers Extraction}
This section describes the method used in tube center extraction.

A typical axial view of phantom in MR images looks like this:
% \begin{figure}[htb]
%   \begin{minipage}[b]{0.8\linewidth}
%     \centering
%     \centerline{\mbox{\includegraphics[width=0.8\linewidth]{}}}
%   \end{minipage}
%   \caption{Axial view}
%   \label{fig:axial_sample}
% \end{figure}

The strategy to extract tube centers is to use one slice to produce a series of masks which specified the tube
locations. For reest of the slices, we use the masks to extract regions where tubes are at and process each
region separately. There are several advantage of doing this:
\begin{itemize}
\item This method will make the whole process runs a lot faster due to significant less amount of pixels needed
  to process.
\item It also has a lot potential to run much more faster. Once 
  a set of masks is produced from a image, not only a number of images following that slice could be processed
  simultaneously, 64 regions in each images could also be processed at the same time after they are extracted.
\item It will be less error prone as well, due to masks narrows down the problem regions.
\end{itemize}

\subsubsection{Mask Extraction}
This section describes how the masks are produced.

The following steps are used to extract a set of tube mask from an axial MR images within the range we have
calculated previously.
\begin{enumerate}
\item Use a threshold to produce a binary image that only contains tube signals. 
  Due to the good suspetibility of $CuSO_4$ solutions, signals from solutions inside MR tubes are much
  higher than other regions of in the same image.
\item Apply canny edge detection on the binary image produced from previous step. Note: Applying canny
  directly on original image with high threshold will also yield a good quality edge images with much
  smaller circles.
\item Masks of tubes at top row, bottom row, left column  and right column are generated first. 
  for top row:
  \begin{enumerate}
  \item Iterate the the top left region of the image, sweeping row wise from left to right to search 
    for a pixel that belongs to left most tube on top row.
  \item Once an edge pixel is found, all the edge pixels of the this tube will be collected, i.e. record and 
    remove from the image. And a center of the tube is computed using steepest descent method. 
  \item Use the center calculated from previous method as a starting point and walk the pixels on the same row
    to the right to obtain 7 other tube centers of top row by repeating previous steps.
  \end{enumerate}
\item Use similar method that obtained centers of top row to obtain bottom rown, left column and right column. 
  For bottom row, lower right region of the image are searched initially, sweeping from right to left, once
  the first tube center is found, algorithm traverse to left to collect the rest tube; 
  for left column, top left region is swept column wise from up to down, after finding the first center, 
  algorithm traverse down ward on the same column to search for other tubes.
  for right column, bottom right region is swept column wise from bottom to top, after finding the first
  center, algorithm traverse upward to search for other tubes.
\item For diagnal tubes, similar initial search is used to look for the first pixel of the first tube. 
  When looking remaining tubes, row wised sweeping is motion is used. 
\item After a center is found, a square region is allocated for a mask for that tube. In total, there are 64
  masks extracted from on image. Each region will have it's top left coordinate on the image as well as 
  width and height.
\end{enumerate}

Center calculation is using steepest descent. Several other algorithms are tested, such as, weighted average,
least square, but steepest descent has the best result among the three.

\subsubsection{Center Extraction Using Masks}

With a set of mask is extracted, it would much easier to extract centers on other images with the masks:
\begin{enumerate}
\item Using each region's top left corner coordinate and it's width and height, 64 much smaller regions only
  containing tubes can be extracted. 
\item Each region can use the same edge algorithm used earlier step, then extract pixels and calculate the
  centers. Center coordinate in the region will add to the top left corner coordinate to become a coordinate
  in original image.
\end{enumerate}


