\subsection{Introduction}
There are total 64 MR tubes filed Copper Sulfate $(CuSO_4)$ places on the outer region of the phantom.
We filled them with $CuSO_4$ due to this solution has very good magnetic suspetibility inside MRI scanner. 
They produce very strong signals in MR images which make them much easier to extract. Tubes can be purchased
separately, and easy to replace if necessary. Tubes are sealed with silicon to prevent leaking and could stored
vertically for a long period of time. 

These tubes are used to estimate MRI scanner's gradient isocenter as well as calculate the x and y axis' 
undistortion parameters. To calculation undistortion parameter, we need both original accurate 3D locations
and distorted 3D locations of each tubes. We can use tube holders' spacing specified in phantom's 
specification to determine their undistorted locations since these holders are precision made. So the missing
information are distorted tubes locations. MRI's distortion is nonlinear, which means the 3D shifts at
different locations are different. The way we extract tubes' distortion is by analyzing the axial 
scan/reconstruction of the phantom to determine a starting and ending slice of those tubes, and for every
slices in between them, we find the center of the each tube. Each bended tube in MR images can be viewed as
series. We will use these cetners in later calculation of undistortion parameters.

\subsection{Range}
In this section, we are discussing how we determine which slice in axial scan to start tube processing and 
which slice to stop tube processing.

Tube holders at both end of the MR tubes will cause the regions of the tubes inside the tube holder to shift. 
Depending on how the original scan is performed, the shifts' direction may vary. If the original scan is
done in axial directin, the the shift will occur inside xy plane; if the original scan is performed in
sagittal direction the shift will occur on z-axis. So this means that we cannot use the signal from the 
regions where tube holders are at. 

One way to determine where the tube holders's z-axis location is to find the middle water tank boundaries. 
Tube holder is usually at least 5mm away from each  tank surfaces in MR images. So after finding the location
of two water tank surfaces we can add or subtract 4mm as two boundaries. For every axial slices, we will
check it's z-axis location to see if that slice is between the boundary, if it is, we will continue to process
the slice, otherwise we will ignore it.

To find a accurate middle tank surfaces, we can work on either coronal or sagittal series, and use the
following steps:

\begin{enumerate}
\item Averge the middle slice of the series with its two adjacent slices to produce a sampling slice. This 
  way we can get rid of some noise and would give a more reliable result.
\item 
\end{enumerate}