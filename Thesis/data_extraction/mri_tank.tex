
\subsection{Introduction}
Naturally due to the similarity of MR image and CT images, one can imagine that we shall reuse the algorithm
we used to estimate water tank surfaces in CT images in MR images. However there are following challenges:
\begin{enumerate}
  \item MR images' quality is not as good as CT images. Especially the water tank appeared at lower portion
    of images. That portion of the phantom is not completely inside the head coil, so the signals from that
    portion of the phantom contains a lot of noises. With that amount of noises, algorithm we used in CT 
    images tends to get incorrect results. % NOTES: Elaborate or add examples 
  \item Unlike CT images, which has no distortion in the images, MR images are distorted. Tank surfaces
    in MR images have curvature. So when processing the curvatures noisy edge, we can't really use straight
    line anymore, there has to be another reliable way to help us trace the broken, noisy curved edges.
  \item Because CT images are not distorted, missing some data points at the edges is not that important.
    With most data we collected, we can still construct a pretty accurate surfaces. However, since MR images
    are distorted, data points on the edges are especially important because they contain the most information
    about the distortion model.
\end{enumerate}

Once approach is to use 

Different cases:
\begin{itemize}
\item Top surface: remove robber pads
\item bottom surface: noise
\item Slices toward the edge % THOUGHT: we could probably combine both saggital and coronal reconstruction to collect a more complete data points
\end{itemize}

NOTE: old get_horizontal_boundaries is obvious not as sophisticated as the newer one, it would not even handle
the mid slice well.

horizontal boundaries for surfaces:
\begin{enumerate}
\item Locate tube regions on either side of the water tank
\item First we need to start from the middle to get an accurate middle slice's left and right boundary. To
  do this we will average middle slice and it's previous and next images to get rid of some of noise signals 
  as well as enhance the real data signals. 
\item We will use both coronal and sagittal set to extract surfaces
\item Bottom tank's boundaries could just use the upper tank's
\item Only the exterior inside surface will use canny for edges, every other surface will use column wised
  data analysis. 
  All surfaces will use the same left and right boundaries derived from exterior inside surface.
  Column wised pixel analysis should be able distinguish padding region and columns that might be outside
  left or right boundaries.
\item Averaging three slices work well for most of the slices. Experiments have shown that 60\% mid slices 
  is a good cut off point
\end{enumerate}