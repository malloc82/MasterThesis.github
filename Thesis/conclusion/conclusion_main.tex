In summary, all major goals of this thesis has been achieved:
\begin{enumerate}
\item A new phantom has been designed and manufactured which accurately captures the distortions inside MR 
  images of a 3T MRI scanner
\item An algorithm was developed to calculate the gradient isocenter of the magnetic field of a 3T MRI scanner. 
  Simulated results have shown that this algorithm worked quite accurately.
\item Several algorithms were also developed to extract required geometry information from CT and MR images 
  for MRI distortion correction.
\item Finally, an algorithm was developed to calculate the distortion correction parameters 
  using the phantom features extracted from the CT and MR images.
\end{enumerate}

In the near future, further tests will be perform to explore the accuracy and limitations of this 
distortion correction algorithm developed in this thesis. There are other sources of geometric distortions in 
MRI images that need to be considered, including magnetic susceptibility distortion and chemical shift.
Lastly, one should compare the performance of the correction algorithm developed in this thesis with that of 
intrinsic distortion correction algorithms that are included in the scanner software.
%  Still Need To be performed to further test the limitation of the this algorithm, e.g. is there
% any region tends to be corrected better than others.